%To compile this chapter separately, comment out the \iffalse and \fi lines

\iffalse
\documentclass[11pt,fleqn]{book} % Default font size and left-justified equations
\usepackage[top=3cm,bottom=3cm,left=3.2cm,right=3.2cm,headsep=10pt,a4paper]{geometry} % Page margins
\usepackage{xcolor} % Required for specifying colors by name
\definecolor{color}{RGB}{128,0,0} % Define the color used for highlighting throughout the book
\definecolor{blue}{RGB}{69,74,159} %Redifines the default for the color blue

%----------------------------------------------------------------------------------------
% Font Settings
\usepackage{avant} % Use the Avantgarde font for headings
%\usepackage{times} % Use the Times font for headings
\usepackage{mathptmx} % Use the Adobe Times Roman as the default text font together with math symbols from the Sym­bol, Chancery and Com­puter Modern fonts
\usepackage{microtype} % Slightly tweak font spacing for aesthetics
\usepackage[utf8]{inputenc} % Required for including letters with accents
\usepackage[T1]{fontenc} % Use 8-bit encoding that has 256 glyphs
\usepackage{csquotes}
\usepackage{endnotes}
\usepackage{setspace}
\onehalfspacing

\usepackage{todonotes}

%----------------------------------------------------------------------------------------
% Bibliography
\usepackage[style=authoryear,
            %sorting=nyt,
            %sortcites=true,
            autopunct=true,
            %babel=hyphen,
            %hyperref=true,
            %abbreviate=false,
            backref=true,
            %refsection=chapter,
            backend=bibtex]
{biblatex}
\addbibresource{../../tese.bib} % BibTeX bibliography file
\defbibheading{bibempty}{}
%bibtex sspanel1-blx

%Another option with natbib which require a different method to do a bibliography section per chapter
%\usepackage[square,numbers,sectionbib]{natbib}
%\usepackage{chapterbib}
%\usepackage{authblk}
%\bibliographystyle{evolution}

%----------------------------------------------------------------------------------------
% Index
\usepackage{calc} % For simpler calculation - used for spacing the index letter headings correctly
\usepackage{makeidx} % Required to make an index
\makeindex % Tells LaTeX to create the files required for indexing

%----------------------------------------------------------------------------------------
\input{../../structure} % Insert the commands.tex file which contains the majority of the structure behind the template

%----------------------------------------------------------------------------------------

\begin{document}
\graphicspath{ {../../Images/}}
\fi


\chapterimage{Headers/chapter_header.pdf}
\chapter{Regular chapter \label{chap:main}}
%\begin{refsection}
%------------------------------------------------
%\listoftodos

\section{Mini table of content in chapter header}
This template includes a small TOC in each chapter. Note that it is NOT formatted so that the content
automatically fits in the box (see chapter \ref{chap:legrand}).


\section{Sections}
\subsection{Subsections}

\begin{itemize}

\item \textbf {\sffamily \footnotesize List of items with text:}
 
Here you can write something about this topic. 

\hspace{15pt} Indentation must be forced if there is a second paragraph within an item.

\item \textbf {\sffamily \footnotesize Another item:}

\lipsum[1]

\end{itemize}

\section{Using citations}

This template uses the package biblatex (\url{https://www.ctan.org/pkg/biblatex})
for reference management. Please mind the compilation instructions in
the main file (``thesis.tex'').

A statement may be followed by single (I love natural selection \parencite{darwin_origin_1859}) or
multiple citations (We love natural selection \parencite{darwin_tendency_1858, darwin_origin_1859}).

References may also appear within a sentence: ``We thank \textcite{legrand_legrand_2013} for much of
this template''. 

Biblatex allows easy customization of parenthetical citations:

\begin{itemize}
  \item \parencite[p. 180]{darwin_origin_1859}.
   \item \parencite[revisão em][]{darwin_origin_1859}.
  \item \parencite[ver][p. 180]{darwin_origin_1859}.
\end{itemize}

\section{Using figures}


The structure.tex file calls the package sidecap (\url{http://www.ctan.org/tex-archive/macros/latex/contrib/sidecap})
which allows captions to be typeset sideways.
The graphic path is defined as ``Images/'' in this same file.


 \begin{figure}
\centering
\includegraphics[width=14cm]{Figures/1-regular-chapter/placeholder.jpg}
 \caption{Regular figure caption}
\end{figure}

\begin{SCfigure}
\centering
\includegraphics[width=6cm]{Figures/1-regular-chapter/placeholder.jpg}
 \caption{
Caption may appear on the side of floats.
 \vspace{8pt}}
\end{SCfigure}


\section{Using tables}

The structure.tex file calls the package tabularx (\url{https://www.ctan.org/pkg/tabularx}) that allows more flexibility in
choosing column widths.


\begin{table}
\centering
\caption{Table caption
}
\begin{tabularx}{\textwidth}{ C{1} | C{1} C{1} C{1} C{1} C{1} C{1} C{1} C{1} C{1} C{1}}
   \hlinewd{1.5pt}
      & \textbf{\textcolor{blue}{A}} & \textbf{\textcolor{blue}{B}} & \textbf{\textcolor{blue}{C}} & \textbf{\textcolor{blue}{D}}
      & \textbf{\textcolor{blue}{E}} & \textbf{\textcolor{blue}{F}} & \textbf{\textcolor{blue}{G}} & \textbf{\textcolor{blue}{H}}
      & \textbf{\textcolor{blue}{I}} & \textbf{\textcolor{blue}{J}}\tabularnewline
   \hline
    \textbf{A} & \textcolor{blue}{7.73/} \newline 5.76 & \vspace{0pt}\textcolor{blue}{7.04} &\vspace{0pt}\textcolor{blue} {7.60}
    & \vspace{0pt} \textcolor{blue}{10.33} &\vspace{0pt}\textcolor{blue}{11.05} & \vspace{0pt} \textcolor{blue}{14.50} 
    &\vspace{0pt} \textcolor{blue}{13.17} & \vspace{0pt}\textcolor{blue}{17.00} &\vspace{0pt} \textcolor{blue}{17.36}
    &\vspace{0pt} \textcolor{blue}{15.78}\tabularnewline
    
    \textbf{B} & \vspace{0pt}6.58 &\textcolor{blue}{6.68/} \newline 6.60 &\vspace{0pt}\textcolor{blue} {7.20}
    & \vspace{0pt} \textcolor{blue}{10.16} &\vspace{0pt}\textcolor{blue}{11.41} & \vspace{0pt} \textcolor{blue}{14.94} 
    &\vspace{0pt} \textcolor{blue}{13.58} & \vspace{0pt}\textcolor{blue}{17.43} &\vspace{0pt} \textcolor{blue}{18.16}
    &\vspace{0pt} \textcolor{blue}{16.04}\tabularnewline
    
    \textbf{C} & \vspace{0pt}8.86 & \vspace{0pt}8.92 &\textcolor{blue}{7.16/} \newline 6.40
    & \vspace{0pt} \textcolor{blue}{10.03} &\vspace{0pt}\textcolor{blue}{10.46} & \vspace{0pt} \textcolor{blue}{13.80} 
    &\vspace{0pt} \textcolor{blue}{12.70} & \vspace{0pt}\textcolor{blue}{15.64} &\vspace{0pt} \textcolor{blue}{16.99}
    &\vspace{0pt} \textcolor{blue}{14.57}\tabularnewline
    
    \textbf{D} & \vspace{0pt}9.52 & \vspace{0pt}9.44 & \vspace{0pt}6.94
    & \textcolor{blue}{10.43/} \newline 6.07 &\vspace{0pt}\textcolor{blue}{10.93} & \vspace{0pt} \textcolor{blue}{14.81} 
    &\vspace{0pt} \textcolor{blue}{13.00} & \vspace{0pt}\textcolor{blue}{14.80} &\vspace{0pt} \textcolor{blue}{16.00}
    &\vspace{0pt} \textcolor{blue}{15.04}\tabularnewline
    
    \textbf{E} & \vspace{0pt}10.16 & \vspace{0pt}10.52 & \vspace{0pt}8.50
    & \vspace{0pt}7.18 &  \textcolor{blue}{10.76/} \newline 6.02 & \vspace{0pt} \textcolor{blue}{14.57} 
    &\vspace{0pt} \textcolor{blue}{12.10} & \vspace{0pt}\textcolor{blue}{14.34} &\vspace{0pt} \textcolor{blue}{16.95}
    &\vspace{0pt} \textcolor{blue}{14.93}\tabularnewline
    
    \textbf{F} & \vspace{0pt}12.56 & \vspace{0pt}13.44 & \vspace{0pt}11.42
    & \vspace{0pt}9.78 &  \vspace{0pt}7.90 & \textcolor{blue}{13.33/} \newline 5.80 
    &\vspace{0pt} \textcolor{blue}{12.42} & \vspace{0pt}\textcolor{blue}{14.10} &\vspace{0pt} \textcolor{blue}{17.00}
    &\vspace{0pt} \textcolor{blue}{15.08}\tabularnewline
    
    \textbf{G} & \vspace{0pt}13.34 & \vspace{0pt}14.62 & \vspace{0pt}12.24
    & \vspace{0pt}10.88 &  \vspace{0pt}8.98 & \vspace{0pt}5.14 
    &\textcolor{blue}{12.27/} \newline 3.93  & \vspace{0pt}\textcolor{blue}{13.00} &\vspace{0pt} \textcolor{blue}{16.44}
    &\vspace{0pt} \textcolor{blue}{14.67}\tabularnewline
    
    \textbf{H} & \vspace{0pt}16.74 & \vspace{0pt}17.54 & \vspace{0pt}14.64
    & \vspace{0pt}13.50 &  \vspace{0pt}10.86 & \vspace{0pt}7.60 
    & \vspace{0pt}6.64   & \textcolor{blue}{10.60/} \newline 5.67 &\vspace{0pt} \textcolor{blue}{14.21}
    &\vspace{0pt} \textcolor{blue}{13.17}\tabularnewline
    
    \textbf{I} & \vspace{0pt}16.74 & \vspace{0pt}17.30 & \vspace{0pt}14.98
    & \vspace{0pt}13.82 &  \vspace{0pt}10.98 & \vspace{0pt}8.10 
    & \vspace{0pt}7.18   & \vspace{0pt}5.98  &\textcolor{blue}{13.64/} \newline 5.31
    &\vspace{0pt} \textcolor{blue}{14.52}\tabularnewline
    
    \textbf{J} & \vspace{0pt}17.36 & \vspace{0pt}8.20 & \vspace{0pt}16.54
    & \vspace{0pt}16.38 &  \vspace{0pt}14.10 & \vspace{0pt}10.46 
    & \vspace{0pt}8.86   & \vspace{0pt}8.18  & \vspace{0pt}7.46
    &\textcolor{blue}{13.40/} \newline 6.47\tabularnewline
    \hlinewd{1.5pt}
\end{tabularx}
\end{table}

\pagebreak

\section{Using boxes}


The ``clearinfobox'' environment can be used to highlight relevant information.
Additionally, this template includes all in-text elements from the Legrand Orange Book
(see chapter \ref{chap:legrand}). The word ``Quadro'' appearing in the box does not change 
automatically when the document language is changed.

\begin{clearinfobox}[Box title]
%Uncomment the following line to make the box appear in the table of contents
%\addcontentsline{toc}{subsection}{\small\sffamily\color{green} Quadro 1.1 - Espécies em anel: conceitos importantes}
%Boxes may include images
 %\includegraphics[width=14cm]{Figures/1-regular-chapter/placeholder.jpg}
 %and text
\lipsum[1]
\end{clearinfobox}

\section{Using footnotes}


There are two types of footnotes:

\begin{itemize}
  \item Short footnotes which appear in the same page \footnote{This is a short footnote}.
  \item Longer footnotes which will appear in a separate page at the end of the document
  and are identified by a letter\footnote{\label{same_number} 
Ver página \pageref{notas}}\endnote {Explanation that will appear in a separate page.}.
\end{itemize}

When there is more than one long footnote in the same page (yes, I do love footnotes) the first
one should be labeled and the additional footnotes should carry the samel label so that they only differ 
in the letter \footnoteref{same_number}\endnote{More info!!}.


%------------------------------------------------
\printbibliography[heading=subbibliography]
%\end{refsection}

\iffalse
\begingroup
\let\enotesize\normalsize
\renewcommand\notesname{Notas finais}
\renewcommand\enoteformat{\noindent[\theenmark]\hspace{0.06in}}
\label{notas}
\theendnotes
\endgroup

\end{document}
\fi

