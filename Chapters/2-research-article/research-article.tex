\iffalse
\documentclass[11pt,fleqn]{book} % Default font size and left-justified equations
\usepackage[top=3cm,bottom=3cm,left=3.2cm,right=3.2cm,headsep=10pt,a4paper]{geometry} % Page margins
\usepackage{xcolor} % Required for specifying colors by name
\definecolor{color}{RGB}{128,0,0} % Define the color used for highlighting throughout the book
\definecolor{blue}{RGB}{69,74,159} %Redifines the default for the color blue

%----------------------------------------------------------------------------------------
% Font Settings
\usepackage{avant} % Use the Avantgarde font for headings
%\usepackage{times} % Use the Times font for headings
\usepackage{mathptmx} % Use the Adobe Times Roman as the default text font together with math symbols from the Sym­bol, Chancery and Com­puter Modern fonts
\usepackage{microtype} % Slightly tweak font spacing for aesthetics
\usepackage[utf8]{inputenc} % Required for including letters with accents
\usepackage[T1]{fontenc} % Use 8-bit encoding that has 256 glyphs
\usepackage{csquotes}
\usepackage{endnotes}
\usepackage{setspace}
\onehalfspacing

\usepackage{todonotes}

%----------------------------------------------------------------------------------------
% Bibliography
\usepackage[style=authoryear,
            %sorting=nyt,
            %sortcites=true,
            autopunct=true,
            %babel=hyphen,
            %hyperref=true,
            %abbreviate=false,
            backref=true,
            %refsection=chapter,
            backend=bibtex]
{biblatex}
\addbibresource{../../tese.bib} % BibTeX bibliography file
\defbibheading{bibempty}{}
%bibtex sspanel1-blx

%Another option with natbib which require a different method to do a bibliography section per chapter
%\usepackage[square,numbers,sectionbib]{natbib}
%\usepackage{chapterbib}
%\usepackage{authblk}
%\bibliographystyle{evolution}

%----------------------------------------------------------------------------------------
% Index
\usepackage{calc} % For simpler calculation - used for spacing the index letter headings correctly
\usepackage{makeidx} % Required to make an index
\makeindex % Tells LaTeX to create the files required for indexing

%----------------------------------------------------------------------------------------
\input{../../structure} % Insert the commands.tex file which contains the majority of the structure behind the template

%----------------------------------------------------------------------------------------

\begin{document}
\graphicspath{ {../../Images/}}
\fi


\chapterimage{Headers/chapter_header.pdf}
\chapter[Research article]{Short title
\label{chap:article}}
%\begin{refsection}
%------------------------------------------------
%\listoftodos

{\Large\sffamily\textbf {Título}}\\
\vspace{0.6cm}
\begin{tikzpicture}[remember picture,overlay]
     \node[white,draw,text=black,text width=11.5cm] at (page cs:0.15,0.1) 
     {\it\footnotesize O conteúdo deste capítulo foi publicado no artigo:
      \textbf{Autores}. ANO. 
      “Título do artigo”. Revista, volume, páginas. doi.      
      };
\end{tikzpicture}
\vspace{0.5cm}
\newline
\abstractseparator
{\large\sffamily\textbf {Resumo}}\\
{Resumo de capítulo/artigo.}\\
\abstractseparator

\section{Specific introduction topic}

\lipsum[1-6]

\section{ETC.}



%\fi
%------------------------------------------------
%\printbibliography[heading=subbibliography]
%\end{refsection}

\iffalse
\begingroup
\let\enotesize\normalsize
\renewcommand\notesname{Notas finais}
\renewcommand\enoteformat{\noindent[\theenmark]\hspace{0.06in}}
\label{notas}
\theendnotes
\endgroup

\end{document}
\fi